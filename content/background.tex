\chapter{Background}

\section{Technologies}

% Variety of terms:
% - Indoor Location-Based Service (ILBS)
% - Indoor localisation
% - Indoor positioning
% - Indroor Positioning Systems (IPS)

% Indoor localization is divided into active and passive systems. In active positioning, the user needs to carry a mobile device that will actively search and collect signals. In passive positioning, the user doesn't need to carry any device. In such systems, the user is detected based on his effects on the propagation of the signal \cite{liu_survey_2020}. 

\subsection{Global Positioning System}

The Global Positioning System (GPS) is the de-facto standard for most positioning systems. It is largely deployed, available worldwide, and embedded in almost all smartphones, watches, cars, etc. But while it can provide a sufficient precision for most outdoor usages, with a precision of around 20m, it is inefficient for indoor localization because of errors by multi-path, Non-Line-of-Sight with the sky \cite{mainetti_survey_2014} and insufficient precision.

% !! TODO: Develop efforts to make it works inside/with a greater precision.

\subsection{QR code}

QR codes are a popular way to provide information to a visitor. However, this passive approach require the visitors to take action in order to get further information, which is not helpful for the majority of the visitors \cite{spachos_ble_2020}. 

\subsection{RFID}

RFID is a widely deployed technology, used by most of the contactless tokens. In the context of indoor localization, it would require a large number of inexpensive tags. It also works without direct LoS \cite{mainetti_survey_2014}. It's a nice alternative to QR-code, and allow a bit more interaction. However, it is also a passive method, that as the same issues than discussed before. It short range make it also more efficient to determine if a user is near the tag rather than having a precise indoor localization \cite{shang_overview_2022}. 

\subsection{NFC}

NFC is a widely deployed technology, available in almost any smartphone nowadays. But as RFID tags, it's communication area is very limited and require a lot of beacons to provide indoor localization. However, it can achieve a very high precision without needing the visitor to carry any special device \cite{cai_museum_2015}. It can be seen as an active alternative to RFID.

%\subsection{Vision}

%Evaluating video data on real time is time, cost, power and resource intensive.

\subsection{Infrared}

Infrared requires a Line of Sight (LoS), does not penetrate through walls, have multi-path errors and require expensive hardware \cite{mainetti_survey_2014}. It has huge requirements on the target environment and is influenced by heat sources and lights \cite{shang_overview_2022}.

\subsection{Ultrasound}

Ultrasound is complex to set up in a large scale, is prone to multi-path errors and is highly sensitive to ambient temperature \cite{mainetti_survey_2014}. It uses the technique of the Time Of Flight and can have an accuracy up to the centimeters, but the real efficiency can be affected by the humidity, the ambient temperature, the air density and the obstacles. It also require a tight synchronization between the devices \cite{shang_overview_2022} \cite{mainetti_survey_2014}.

\subsection{Ultra Wide Band}

Ultra wide band can provide a very accurate localization, based on the Time-Of-Arrival techniques. It's also power efficient, has a fine resolution and is robust in harsh environments. However, it requires a lot of extra hardware devices \cite{spachos_ble_2020} and is expensive \cite{shang_overview_2022}.

\subsection{Wi-Fi}

The Wi-Fi is one of the most used systems for indoor localization, as it's the most widely deployed indoor infrastructure and thus can partially rely on existing infrastructure. It's also cost effective \cite{mainetti_survey_2014} and do not require extensive knowledge for users or maintainers \cite{shang_overview_2022}. The users don't have to carry any special device, except their own smartphone. However, the devices are heterogeneous and may differ widely from the reference device(s) used for initial setup \cite{liu_survey_2020}. It doesn't require a direct LoS \cite{mainetti_survey_2014}, even if the environment may have a huge impact on the precision, range and multi-path effect \cite{liu_survey_2020}. Most used techniques are Cell of Origin method, triangulation and RSS-based fingerprinting. It can achieve a theoretical precision of a few centimeters in a dense, errorless and open environment but usually achieve a precision of a few meters for more realistic ones \cite{liu_survey_2020}. More specifically, RSSI is the current mainstream system but is prone to noise and interfaces in a dense area \cite{spachos_ble_2020}. The precision can be adjusted based on the density \cite{shang_overview_2022}. It offer a proper balance between efforts and accuracy \cite{ali_locali_2017}. There is two main methods:
 RSSI heat maps, that allow to visually describe the infrastructure, detects it's weaknesses and use simpler algorithms than most others \cite{ali_locali_2017} and RSSI fingerprinting, that compare user values to a database of registred reference values. 

\subsection{Bluetooth}

Bluetooth is as deployed as Wi-Fi, and even more in mobile devices such as smartphone, watches, headphones, etc. There is an enormous amount of Bluetooth beacons available on the market, for very cheap \cite{spachos_ble_2020} and they are easier to modulate to reduce or extend their range. Typical user devices have a range of 10 to 15 meters \cite{mainetti_survey_2014}. In real case scenarios, it provides an accuracy close to Wi-Fi, from 2m to 3m \cite{mainetti_survey_2014} \cite{spachos_ble_2020}. Bluetooth also require less power than Wi-Fi, because it runs in a special \textit{low power} mode. Bluetooth Low Energy (BLE) also provides more privacy, because their beacons only send a signal, and are not listening \cite{spachos_ble_2020}. The receiver has to handle the data, and can do all the work locally. Signals can be send every 20ms - 10s depending on the usage, and have a huge impact on battery life of the beacons \cite{spachos_ble_2020}, especially if they run on battery. While battery is often used because of the ease of deployment and that a beacon can work for months with just a coin cell, the infrastructure could be more error-prone and the maintenance burden would still be high \cite{spachos_ble_2020}. The precision depends on the distance between the Beacon and the receiver. While the literature is uncertain about the optimal density of beacons, it's accepted that it has a huge impact on the accuracy and require environment-specific tests \cite{spachos_ble_2020} \cite{shang_overview_2022}. BLE beacons can reach up to 60 meters, but it has a huge impact on the power consumption. For those reasons, a transmission range of 2-5m is usually considered \cite{spachos_ble_2020}. But it could have an impact on indoor localization while not exactly in front of art pieces. Bluetooth don't interfere with other wireless infrastructures that may already be present in the area such as Wi-Fi, GPS or FM. However, they can interfere with other Bluetooth devices \cite{spachos_ble_2020}  and maybe even more in a digital museum, as they could be used to listen to audio. In a perfect environment, the theoretical RSSI with BLE is defined by equation \ref{eq:BLE_RSSI} where $A$ is the received signal strength at 1m, $n$ the signal propagation constant, depending mainly on the environment and $d$ the distance. As we can see, the RSSI is logarithmically dependent of the distance and explain why the precision decrease with the distance \cite{spachos_ble_2020}.

\begin{equation} \label{eq:BLE_RSSI}
    RSSI = A - 10n \cdot \log_{10}d
\end{equation}

\subsection{Hybrid}

More and more systems combine more than one method to get the best possible accuracy \cite{shang_overview_2022}. Various device sensors can also be used to improve the real-time precision between the emission of the beacons such as inertial senors, accelerometers, gyroscopes, etc \cite{ali_locali_2017}. 

\subsection{Summary}

Espresence

\section{Market products}

\begin{itemize}
    \item Mapsted. 3-5m accuracy.
    \item Steerpath (used by Aalto). 2-5m accuracy.
\end{itemize}

\subsection{Google Indoor Map}
