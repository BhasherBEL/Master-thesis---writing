\chapter{Introduction}

Most of the systems allows to captures useful analytics about the visitors' retention time \cite{spachos_ble_2020}.

Indoor localization is a vast subject, covered widely in the papers, and with a lot of different approach. While some papers use the existing infrastructure, such as the Wi-Fi or the FM and the user's devices, others require a specific installation or don't use the user's device. Even if this factor is not widely discussed in the technical papers, they may have a non-negligible environmental and social impact.

There is many indoor localization systems, and they all have their pro and cons depending on their application context. Some may require installation of new hardware, while other use the already available infrastructure \cite{ali_locali_2017}. Some require the user to carry a specialized device, and others can rely only on the users' smartphone, or even on the presence of the user itself, without any device.

Indoor localization depends on trade-off among performance parameters, user requirements and environmental conditions \cite{mainetti_survey_2014}.

% Environmental impact: Energy and material consumption.
% Social impact: What if it requires a BLE smartphone?