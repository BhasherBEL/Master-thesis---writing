\chapter{Introduction}
\label{chap:introduction}

\section{Motivations}

% Why IPS in a Museum?
% Which museums issues, technology can help fixing

% Passive approach is meh
% Not accessible
% Users acceptance

Over the past decade, museums have faced increasing competition and financial challenges due to decreased government funding, pushing many to adopt a more consumer-oriented approach \cite{chiappa_emotions_2014}. Information and communication technologies (ICT) have become essential in enhancing visitor experiences, making exhibits more interactive and engaging. When designed intuitively, such technologies can improve cognitive engagement and positively impact learning outcomes \cite{pallud_impact_2017}.   

Indoor localization presents a promising solution by enabling seamless navigation and interaction with exhibits, enriching the visitor experience. Additionally, it allows museums to collect valuable data to optimize exhibit layouts and visitor flows. However, challenges remain, including concerns about social acceptability and reduced interaction between visitors \cite{tom_dieck_enhancing_2018}. Furthermore, the environmental impact of indoor localization systems, such as energy consumption and hardware requirements, must be carefully considered.

Balancing engagement, sustainability, and social interaction is crucial for museums adopting new technologies. Indoor positioning systems, if well-designed with sustainability in mind, such as using energy-efficient devices, minimizing infrastructure demands and avoid extensive algorithm training, can enhance accessibility and personalization while fostering a more integrated and sustainable visitor experience.

\section{Introduction to indoor localization}

% What's IPS

Indoor localization is a widely researched field with diverse approaches, many of which enable the collection of valuable analytics, such as visitor retention time \cite{spachos_ble_2020}. Some methods leverage existing infrastructure, such as Wi-Fi or FM signals and users' devices, while others require dedicated installations or operate independently of user devices \cite{ali_locali_2017}. Each approach involves trade-offs in terms of accuracy, hardware requirements, and environmental impact, an often-overlooked factor despite its potential significance \cite{mainetti_survey_2014}.   

The terminology varies, with \textit{Indoor Positioning Systems (IPS)}  referring to the technology, \textit{indoor positioning} describing the process, and \textit{Indoor Location-Based Services (ILBS)}  using location data for added functionalities. Finally, \textit{Indoor localisation} is simply a spelling variant. 

\section{Research objectives and organization}

% Why do I do what I want to do?
% Chapter orders

% Environmental impact: Energy and material consumption.
% Social impact: What if it requires a BLE smartphone?

This research aims to develop an indoor positioning system for the Musée L of Louvain-la-Neuve to enhance visitor experience while considering environmental and social impacts. A key goal is to improve navigation and accessibility by designing a user-friendly system that enhances engagement and knowledge retention.  

To achieve this, the study will evaluate various indoor localization technologies, assessing their accuracy, infrastructure compatibility, and sustainability. Particular attention will be given to energy consumption, hardware requirements, and privacy concerns. The necessary background information will be presented in \autoref{chap:background}, while existing solutions will be analyzed in \autoref{chap:related-work}.   

A prototype tailored to Musée L will be developed and tested, with a focus on reliability, ease of deployment, and minimal disruption to exhibits. In addition, an Android application will be designed to leverage the location information, enhancing visitor experience through personalized guidance and contextual information. The research methodology will be outlined in \autoref{chap:methodology}, and the implementation details will be explained in \autoref{chap:implementation}.   

The system’s performance will be evaluated through technical and user-based assessments, analyzing visitor movement patterns and engagement. The results and their implications, including limitations, will be discussed in \autoref{chap:discussion}. Finally, conclusions and future research directions will be provided in \autoref{chap:conclusion}. This study aims to create an effective and sustainable indoor localization solution for Musée L while offering a framework applicable to other cultural institutions. 