\chapter{Discussion and Limitations}
\label{chap:discussion}

This chapter provides a comprehensive evaluation of the proposed localization system for museum environments, combining experimental insights with a candid acknowledgment of the study’s limitations. By analyzing the system’s performance, scalability, privacy measures, and sustainability, we aim to contextualize its potential within the broader landscape of indoor localization technologies and it's application in the specific context of museums. Additionally, we explore the constraints imposed by hardware, environmental factors, and user behavior, offering a balanced perspective on the system’s strengths and areas for improvement.

\section{Localization Accuracy}
The localization accuracy of the system was mainly evaluated within a single museum environment, yielding median errors of 2.8 meters after applying a Kalman filter. This precision allows the system to be used for general proximity tracking but is not sufficient for application requiring relatively precise localization, limiting its applications. The final design of the mobile application had to take those challenges into account. The usage choice for the application to detect which area of the museum the user is in, and provide a by-section audio general explanation while following a recommended visit order is a good application of such constraints, while a per-art piece detection and audio description would probably not be appropriate, and thus have not been chosen.

At the same time, while out of scope of the current study, there is no guarantee that the visitor would especially prefer a by-art piece experience, looking at the way most visitors visits museums. Many IT systems in museums fail to convince the visitor or to improve the user experience due to inadequacy between system and most visitors expectations. %\textbf{SOURCE NEEDED}.

Another important aspect is the precision and limitations of area detection. Experimental results showed that the system correctly identified the user's area in over 80\% of cases. A near-perfect accuracy was observed in areas where beacon placement allowed for small and well-balanced triangulation and for minimal architectural interference. In contrast, detection performance declined in more complex spaces, with off-centered beacons and physical obstructions. These findings highlight that the effectiveness of area detection is closely tied to the spatial configuration of the environment and the quality of beacon deployment, suggesting that improvements in these domains are essential for achieving consistently high precision throughout the entire space.

\section{System Responsiveness}

System responsiveness was another critical factor, as delays in updating visitor positions or delivering information could negatively impact the user experience. Initial tests suggested that the system performs adequately in real-time scenarios, but further optimization is needed to ensure consistency, especially in environments with high signal interference.

A key challenge encountered during development was the trade-off between real-time responsiveness and localization precision. Achieving higher precision often requires more extensive data aggregation, which can introduce latency and reduce the immediacy of position updates. Conversely, prioritizing real-time updates may necessitate using less data or simpler algorithms, potentially reducing accuracy. In this system, a balance was sought to ensure that position updates were timely enough to be useful for guiding visitors, while still maintaining acceptable localization accuracy. This trade-off influenced both the choice of filtering techniques and the design of the user experience, favouring general area detection and timely feedback over fine-grained, but potentially delayed, localization.

\section{Experimental Constraints}
The study was conducted on a single floor of the museum, with a limited number of plug sockets available for beacon deployment. This constraint restricted the density and placement of beacons, potentially affecting localization accuracy. The experimental results tend to indicate that the beacon density was just sufficient to have interesting results, but that a higher density would significantly improve the results and the opportunities for the system. However, such a density was not feasible within the study's environment.

Additionally, the controlled environment of a single museum limits the generalizability of the results. Testing in diverse settings, such as on multiple floors, would provide a more comprehensive evaluation but was beyond the scope of this study.

\section{Hardware and Software Limitations}
The system relied exclusively on ESP32 devices as beacons, which, while cost-effective and energy-efficient, may not represent the full range of hardware options available. Different beacon types could yield different results in terms of accuracy and range.

The application was developed using Flutter, a framework designed for cross-platform compatibility. However, testing was limited to Android devices due to hardware availability, leaving unexplored the potential challenges of iOS or other platforms. Additionally, the system’s reliance on pre-packaged art piece data ensures offline functionality but requires users to download updates for exhibit changes, which could hinder usability over time. Relying on the visitor phone may lead to performance variability due to differences in hardware and software configurations. 


Finally, both Android and iOS impose strict limitations on the frequency and duration of Bluetooth scans, primarily for privacy protection and energy efficiency. These operating system-level restrictions are a major constraint for any real-time localization system relying on Bluetooth, as they directly limit how often the application can detect nearby beacons and update the user's position. On Android, background scan intervals are throttled and may be delayed or batched, while on iOS, scanning is even more restricted, especially when the app is not in the foreground. This means that, regardless of hardware or algorithmic improvements, the achievable update rate and thus the responsiveness of the system are fundamentally capped by these policies.

As a result, these limitations have a significant impact on the overall performance and user experience of the system, and must be considered a primary factor in the design and evaluation of any Bluetooth-based indoor localization solution. In contrast, using dedicated devices for localization, rather than relying on visitor smartphones, could bypass these restrictions, allowing for much higher scan frequencies and more precise, real-time localization. This would enable substantial improvements in both accuracy and responsiveness, but would come at the cost of increased hardware deployment and maintenance requirements.

\section{Environmental Factors}
The dense arrangement of art pieces on the museum floor introduced significant signal obstructions, challenging the system’s ability to maintain consistent accuracy. While this environment provided valuable insights into the system’s robustness, it also highlighted the need for adaptive strategies, such as advanced signal processing techniques, to mitigate the impact of environmental factors and raised the limitation of using beacons that require a plug rather than battery-based beacons.

\section{Scalability and Management}
The system was designed with scalability in mind, allowing for the potential expansion of the beacon fleet as needed. However, managing many beacons could become logistically challenging, particularly in terms of installation, maintenance, and synchronization. Implementing a dedicated management system, such as a centralized control platform for beacon configuration and monitoring, could address these challenges, though it was not explored in this study.

\section{Ethical and Privacy Considerations}
Privacy and ethical concerns were addressed through anonymized data collection and visitor consent protocols. Compliance with GDPR regulations was ensured by avoiding the use of Personally Identifiable Information (PII). While these measures are essential for maintaining visitor trust and legal adherence, they may limit the system’s ability to offer personalized experiences. Future work could explore privacy-preserving technologies to balance data utility with user privacy.

\section{Sustainability Impact}
The system’s sustainability was evaluated across both hardware and software dimensions. The use of small, energy-efficient beacons and visitor smartphones reduced the need for dedicated devices, minimizing resource consumption. Local computation on the smartphone further enhanced energy efficiency by eliminating the need for cloud-based processing. However, a detailed lifecycle analysis, including measurements of the power usage of the different devices and infrastructures, was beyond the scope of this study. Future evaluations could incorporate metrics such as carbon footprint and material recyclability to provide a more holistic understanding of the system’s environmental impact.

\section{User Interaction Variability}
The system was designed to predict and accommodate visitor behaviour, but deviations from predefined patterns may limit its effectiveness. For example, visitors who choose to explore the museum in non-linear or unpredictable ways may not fully benefit from the system’s features. Despite this, the design ensures that visitors can explore freely without interference, preserving the core museum experience. Future iterations could incorporate more adaptive user interaction models to better accommodate diverse visitor behaviours. 

% -> The user may listen and move

 \section{Implication for the Museums}
 Implementing an indoor localization system in museums introduces several practical constraints that must be addressed. Infrastructure challenges, such as limited power outlets or dense exhibit layouts, can hinder beacon deployment and signal coverage. Older or historically significant buildings may lack the necessary infrastructure, while multi-floor or complex spaces amplify these difficulties. Additionally, managing a large fleet of beacons can strain museum resources, requiring ongoing maintenance and synchronization. Financial constraints may further limit adoption, particularly for smaller institutions with limited budgets. Finally, ensuring compliance with privacy regulations and addressing visitor concerns about data use are essential for building trust. These constraints highlight the need for careful planning and resource allocation to ensure the system’s feasibility and effectiveness in real museum environments.
