% \chapter{Conclusion and future work}
% \label{chap:conclusion}

% Future work: Data fusion \cite{barsocchi_detecting_2021}.

% Bluetooth 5.1 allows AoA and AoD \cite{barsocchi_detecting_2021}.

% 3D

\chapter{Conclusion and Future Work}
\label{chap:conclusion}

\section{Conclusion}
This study explored the design, implementation, and evaluation of an indoor localization system tailored for museum environments, specifically deployed on the sixth floor of Musée L. The system demonstrated its potential to improve visitor navigation and engagement by offering contextual and spatially aware content through a user-friendly mobile application. 

Key technical findings include a median localization error of 2.8 meters, with over 75\% of estimates falling within 3.86 meters, sufficient for room-level accuracy. Signal filtering methods such as Kalman filtering contributed to stabilizing noisy signal data and improved the consistency of location estimates. Area detection accuracy exceeded 80\%, validating the system’s ability to correctly assign visitors to specific sections. Nonetheless, occasional signal dropouts underscore the critical role of optimized beacon coverage and signal quality in complex indoor settings. 

The system was designed to be energy-efficient by leveraging visitors’ smartphones and lightweight ESP32 beacons. Data privacy was respected through anonymized data handling, aligning with ethical standards for digital installations in public cultural spaces. 

The study also confirmed the viability of Bluetooth-based solutions for indoor localization in museum contexts, while acknowledging inherent limitations, such as variability in signal propagation and hardware constraints related to power supply deployment. Additionally, user experience remains a critical dimension, as passive guidance systems must accommodate diverse visitor expectations and behaviours. 

Overall, the system provides a foundational step toward more engaging, accessible, and intelligent museum visits. It shows the feasibility of scalable integration of indoor positioning technologies, with promising avenues for enhancement.

\section{Future Work}

Future research could expand the current system’s capabilities by incorporating data fusion techniques, integrating signals from multiple sources such as Wi-Fi, cellular networks, and inertial sensors. By leveraging the complementary strengths of these technologies, the system could achieve increased localization accuracy and resilience, particularly in challenging environments with dense exhibit layouts or significant signal interference. Such multimodal integration would enable more consistent tracking experiences across a broader spectrum of visitor devices and movement patterns. 

Another promising research direction is the adoption of Bluetooth 5.1 features, specifically Angle of Arrival (AoA) and Angle of Departure (AoD). These technologies introduce directional information into the localization process, substantially enhancing spatial resolution and reducing ambiguity caused by signal reflections or multipath effects. Implementing AoA/AoD could make the system more robust in architecturally complex or compartmentalized museum settings, enabling more precise content triggering and visitor guidance. 

In addition, extending the system to support three-dimensional (3D) localization would significantly improve applicability in multi-floor or vertically layered exhibition spaces. This would necessitate accounting for floor structures, vertical signal attenuation, and non-uniform ceiling heights, but would greatly enrich the tracking experience and facilitate new kinds of spatial interactions with content. 

To improve sustainability and deployment scalability, future work should also examine intelligent beacon management and energy-efficient hardware strategies. Solutions may include centralized platforms for monitoring and updating beacon networks, dynamic transmission scheduling, or the use of self-sustained beacon hardware such as solar-powered modules. These innovations could reduce maintenance overhead and support long-term integration into museum infrastructures. 

Altogether, these future directions offer critical pathways to enhance the technical performance, adaptability, and long-term sustainability of indoor localization systems, and could position them as essential tools in the evolution of intelligent and inclusive museum experiences.
