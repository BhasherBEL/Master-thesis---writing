% \chapter{Conclusion and future work}
% \label{chap:conclusion}

% Future work: Data fusion \cite{barsocchi_detecting_2021}.

% Bluetooth 5.1 allows AoA and AoD \cite{barsocchi_detecting_2021}.

% 3D

\chapter{Conclusion and Future Work}
\label{chap:conclusion}

\section{Conclusion}
This study explored the development and evaluation of an indoor localization system tailored for museum environments. The system demonstrated its potential to enhance visitor experiences by providing seamless navigation and contextual information through a mobile application. Key findings include the system’s ability to achieve median localization errors of 1.5 to 2.5 meters, its energy-efficient design leveraging visitor smartphones and small beacons, and its compliance with privacy regulations through anonymized data collection. While challenges remain, such as signal obstructions in dense exhibit layouts and the need for scalable beacon management, the system represents a promising solution for museums seeking to integrate advanced technologies into their visitor engagement strategies.

\section{Future Work}
Future research could explore data fusion techniques to integrate information from multiple sources, such as Wi-Fi, cellular signals, and inertial sensors. This approach could significantly improve localization accuracy and robustness, particularly in environments with high signal interference or dense exhibits. By combining complementary data streams, the system could achieve more reliable and precise tracking.

Another promising direction is the adoption of Bluetooth 5.1’s Angle of Arrival (AoA) and Angle of Departure (AoD) features. These technologies enable the use of directional information to enhance indoor positioning precision, reducing errors caused by signal reflections or obstructions. Implementing AoA/AoD could make the system more adaptable to the architectural complexities of museums, such as multi-room layouts or spaces with reflective surfaces.

Extending the system to support three-dimensional localization would further expand its applicability. This capability would allow for precise tracking in multi-floor museum environments, addressing a significant limitation of current two-dimensional approaches.

Finally, scalable beacon management and energy optimization strategies could enhance the system’s feasibility for broader adoption. Developing centralized management platforms or exploring alternative hardware solutions, such as solar-powered beacons, could reduce operational challenges and improve sustainability. These future directions aim to address current limitations and unlock the system’s full potential in museum settings.